\documentclass[onecolumn]{article}
\usepackage{xeCJK}
\usepackage{amsmath}
\usepackage{listings}
\usepackage{xcolor}
\usepackage{hyperref}
\setlength{\parindent}{0pt}
\renewcommand{\baselinestretch}{1.0}
\lstset{
	frame=tb, % draw a frame at the top and bottom of the code block
	showstringspaces=false, % don't mark spaces in strings
	numbers=left, % display line numbers on the left
	commentstyle=\color{green}, % comment color
	keywordstyle=\color{blue}, % keyword color
	stringstyle=\color{red} % string color
}
\usepackage[a4paper,left=20mm,right=20mm,top=15mm,bottom=15mm]{geometry}  


\begin{document}
1 令$n=2^{a}3^{b}5^{c}$,它的因子个数为$k=(a+1)(b+1)(c+1)$。所以$k=1,2,3,4,5,6$时对应的$n=1,2,4,6,16,12$ \par
~\\
2 $Gcd(n,m)*Lcm(n,m)=n*m$.因为对于某个素数$p$,$m,n$中$p$的个数的最小值最大值分别在最大公约数和最小公倍数中\par
$Gcd((n)mod(m),m)*Lcm((n)mod(m),m)=(n)mod(m)*m$\par
$Gcd(n,m)=Gcd((n)mod(m),m)$ \par
$\Rightarrow Lcm(n,m)=Lcm((n)mod(m),m)*\frac{n}{(n)mod(m)}$\par
~\\
3 $x$是整数时满足,$x$为实数时$\pi (x)-\pi(x-1)=[\left \lfloor x \right \rfloor is$ $prime]$\par
~\\
4 depth1: $\frac{1}{1},\frac{1}{-1},\frac{-1}{-1},\frac{-1}{1}$\par
depth2: $\frac{1}{2},\frac{2}{1},\frac{2}{-1},\frac{-1}{-2},\frac{-2}{-1},\frac{-2}{1},\frac{-1}{2}$ \par
如果把分子分母看作一个二维向量的话,每一层都是顺时针排列的。 \par
~\\
5 \par
$L^{k}=\begin{bmatrix}
1 & k\\
0 & 1
\end{bmatrix}$ \par
$R^{k}=\begin{bmatrix}
1 & 0\\ 
k & 1
\end{bmatrix}$ \par
~\\
6 $(x)mod(0)=x\rightarrow a=b$ \par
~\\
7 $m$需要满足$(m)mod(10)=0,(m)mod(9)=k,(m)mod(8)=1$  \par
$(m)mod(10)=0$说明$m$是偶数,$(m)mod(8)=1$说明$m$是奇数。这是矛盾的。\par
~\\
8 $9x+y=3k,10x=5p$.这说明$y$可以取0,3,$x$可以取0,1.\par
~\\
9 $3^{2t+1}mod(4)=3$。所以$3^{2t+1}=4k+3$.所以$\frac{3^{2t+1}-1}{2}=2k+1$是奇数。\par
另外$\frac{3^{77}-1}{2}$可以被$\frac{3^{7}-1}{2}$整除。因为$3^{77}-1=(3^{7}-1)(3^{70}+3^{63}+..+3^{7}+3^{0})$\par
~\\
10 $999=3^{3}37^{1}\rightarrow \varphi (999)=999(1-\frac{1}{3})(1-\frac{1}{37})=648$\par
~\\
11 $f(n)=g(n)-g(n-1)\rightarrow \sigma (0)=1,\sigma (1)=-1,\sigma (n)=0,n>1$\par
~\\
12 $\sum_{d|m}\sum _{k|d}\mu (k)g(\frac{d}{k})=\sum_{d|m}\sum _{k|d}\mu  (\frac{d}{k})g(k)=\sum_{k|m}\sum _{d|\frac{m}{k}}\mu (d)g(k)=\sum_{k|m}g(k)*[\frac{m}{k}=1]=g(m)$\par
~\\
13 $n$的每个质因子个数都是1.(1)$n_{p}\leq 1$ (2) $\mu (n)\neq 0$ \par
~\\
14 $k>0$时两个都成立。\par
~\\
15 很明显5不是任何$e_{n}$的因子。首先对于模5来说,$e_{1}=2,e_{n}=e_{n-1}^{2}-e_{n}+1$,所以这个模的结果依次是2,3,2,3,不会出现0.\par
~\\
16 $\frac{1}{e_{1}}=\frac{1}{2},\frac{1}{e_{1}}+\frac{1}{e_{2}}=\frac{5}{6},\frac{1}{e_{1}}+\frac{1}{e_{2}}+\frac{1}{e_{3}}=\frac{41}{42}$,由此猜测$\sum_{i=1}^{k}\frac{1}{e_{i}}=\frac{e_{k+1}-2}{e_{k+1}-1}$\par
~\\
假设前$n$项都成立,即$\sum_{i=1}^{n}\frac{1}{e_{i}}=\frac{e_{n+1}-2}{e_{n+1}-1}$ \par
~\\
那么$\sum_{i=1}^{n+1}\frac{1}{e_{i}}=\frac{e_{n+1}-2}{e_{n+1}-1}+\frac{1}{e_{n+1}}=\frac{(e_{n+1}-1)e_{n+1}-1}{(e_{n+1}-1)e_{n+1}}=\frac{e_{n+2}-2}{e_{n+2}-1}$ \par
~\\
17 $Gcd(f_{m},f_{n})=Gcd(f_{m},(f_{n})mod(f_{m}))=Gcd(f_{m},2)=1$ \par
~\\
18 如果$n=rm$且$r$为奇数,那么有$2^{n}+1=(2^{m}+1)(2^{n-m}-2^{n-2m}+2^{n-3m}-...+1)$,比如$2^{12}+1=(2^{4}+1)(2^{8}-2^{4}+1)$ \par
~\\
19 $\left \lfloor \frac{\varphi (k+1)}{k} \right \rfloor=1$当且仅当$k+1$为素数。所以第一个式子表示$[2,n]$中素数的个数,即$\pi (n)$ \par
第二个式子$\sum_{1\leq k<m}\left \lfloor \frac{\frac{m}{k}}{\left \lceil \frac{m}{k} \right \rceil} \right \rfloor$当且仅当$m$为素数时等于1,否则大于1。所以也表示$\pi (n)$ \par
$((k-1)!+1)mod(k)=0$当且仅当$k$为素数。所以也表示$\pi (n)$ \par
~\\
~\\
20 $p_{1}=2$。令$p_{n}$是满足大于$2^{p_{n-1}}$的最小素数,那么有$2^{p_{n-1}}<p_{n}<2^{1+p_{n-1}}$,那么$b=lim_{n\rightarrow oo}lg^{(n)}p_{n}$ \par
~\\
~\\
21 由上面的题目20可以得到$p_{n}<10^{n}$.证明如下\par
\begin{itemize}
	\item 首先$n=1$时满足,有$2<10$
	\item 在$(10^{n},2*10^{n}]$之间一定存在一个素数,所以$p_{n+1}\leq 2*10^{n}<10^{n+1}$
\end{itemize}
因此$K=\sum_{k\ge 1}\frac{p_{k}}{10^{k^{2}}}=\frac{2}{10}+\frac{3}{10^{4}}+\frac{5}{10^9}+...$. \par
~\\
22 假设含有$t$个1.$(111..11)_{b} = \frac{b^{t}-1}{b-1}$。如果$t$不是素数,设$t=nm$,那么$\frac{b^{t}-1}{b-1}=\frac{b^{mn}-1}{b-1}=\frac{b^{m}-1}{b-1}*(b^{nm-m}+b^{nm-2m}+...+1)$ \par
~\\
23 $\rho (2k+1)=0,\rho(2k)=\rho(k)+1$. \par
假设盘子的编号为$0,1,2,...,n-1$,第$k$次移动的盘子编号为$\rho(k)$,可以用数学归纳法证明\par
~\\
24 假设$n=\sum_{k=0}^{m-1}d_{k}p^{k}(0\le d_{k}<p)$\par
那么第$k$位对$\varepsilon _{p}(n!)$的贡献为$d_{k}(1+p+..+p^{k-1})=\frac{d_{k}(p^{k}-1)}{p-1}$,累加所有项可以得到$\varepsilon _{p}(n!)=\frac{n-\nu _{p}(n)}{p-1}$ \par
~\\
25 (1)a成立: 有$m\setminus \setminus n\leftrightarrow m_{p}=0 || m_{p}=n_{p}$.另外$m,k$互质,则对应的素数因子互不影响\par
(2) 在$n=12,m=18$时b不成立 \par
~\\
26 是的,因为$G_{n}$是Stern-Brocot的一个子树。因为如果一个Stern-Brocot的结点属于$G_{n}$,那么这个结点的两个父节点也属于$G_{n}$,并且他们是小于和大于这个结点的结点中与这个节点最靠近的。\par
~\\
27 首先如果两个字符串一样长,那么只需要按照字符串比较大小即可。否则,可以在较短的一个串后面补字符M直到长度相等然后按照字符串大小比较即可。补M是因为一个结点左孩子都小于当前结点,右孩子都大于当前结点,而M正好满足$L<M<R$ \par
~\\
28 $\frac{1}{0},\frac{1}{1}$ \par
$R^{3}:\frac{2}{1},\frac{3}{1},\frac{4}{1}$ ,每次加上$\frac{1}{0}$ \par
$L^{7}:\frac{7}{2},\frac{10}{3},\frac{13}{4},\frac{16}{5},\frac{19}{6},\frac{22}{7},\frac{25}{8}$,每次加上$\frac{3}{1}$ \par
就这样,下一行的分子分母的公差为上一行倒数第二个数字的分子分母,因为那个是它的左祖先 \par
~\\
29 对于$[0,1)$中的数字$x$来说,$1-x$的二进制就是$x$的二进制表示中将01交换,因为$1=\sum_{k>0}\frac{1}{2^{k}}$.那么对于$(0,\infty )$$中的数字\alpha$来说,交换$LR$就是$\frac{1}{\alpha}$。因为Stern-Brocot中对称的数字恰好是互为倒数。所以$1-x$对应于$\frac{1}{\alpha}$\par
~\\
30 $[A,A+m)$中的数字$x$模$m$各不相同,所以$r$元组$((x)(mod)(m_{1}),(x)(mod)(m_{2}),..,(x)(mod)(m_{r}))$各不相同。所以总有一个元组是$((a_{1})(mod)(m_{1}),(a_{2})(mod)(m_{2}),..,(a_{r})(mod)(m_{r}))$ \par
~\\

31 $(b)mod(d)=1\rightarrow (b^{m})mod(d)=((kd+1)^{m})mod(d)=1$\par
所以$((a_{m}a_{m-1}...a_{1}a_{0})_{b}=\sum_{k=0}^{m}a_{k}b^{k})mod(d)=\sum_{k=0}^{m}a_{k}$ 也就是说,只要$(b)mod(d)=1$,那么一个$b$进制的数字能够被$d$整除当且仅当各位数字之和能够被$d$整除\par
~\\
32 假设$n\perp m$.那么下面两个集合相等. \par
$\left \{ (kn)mod(m)|k\perp m ,1\le k <m\right \}=\left \{ k|k\perp m ,1\le k <m \right \}$\par
所以将两边的$\varphi (m)$个数字乘起来,两边除以$\prod _{k\perp m ,0\le k <m}k$即可 \par
~\\
33 $h(1)=1$,假设$n\perp m$,那么$h(mn)=\sum_{d\perp mn}f(d)g(\frac{mn}{d})=\sum_{x\perp m,y\perp n}f(xy)g(\frac{m}{x}\frac{n}{y})=\sum_{x\perp m}\sum_{y\perp n}f(x)g(\frac{m}{x})f(y)g(\frac{n}{y})=h(n)h(m)$ \par
~\\

34 在公式4.56中,如果$d$不是整数,那么$f(d)=0$,所以$g(m)=\sum_{d|m}f(d)=\sum_{d|m}f(\frac{m}{d})=\sum_{d\geq 1}f(\frac{m}{d})$ \par
~\\
35 下面使用的符号与公式4.5相关的符号相同。 \par
$m^{'}=\overline{m}-\left \lfloor \frac{n}{m} \right \rfloor\overline{r},n^{'}=\overline{r}\rightarrow I(m,n)=m^{'}=I(m,\overline{r})-\left \lfloor \frac{n}{m} \right \rfloor I(\overline{r},m)=I(m,(n)mod(m))-\left \lfloor \frac{n}{m} \right \rfloor I((n)mod(m),m),I(n,m)=n^{'}=(n)mod(m)$ \par
$I(0,n)=0,I(m,0)=1$ \par
~\\
36 首先证明2不可以。\par
假设2可以分解为两个非单位数乘积,即$2=(a+b\sqrt{10})(c+d\sqrt{10})=(ac+10bd)+(ad+bc)\sqrt{10}\rightarrow ac+10bd=2,ad+bc=0\rightarrow (a-b\sqrt{10})(c-d\sqrt{10})=(ac+10bd)-(ad+bc)\sqrt{10}=2\rightarrow (a^{2}-10b^{2})(c^{2}-10d^{2})=4$\par
由于$|a^{2}-10b^{2}|\neq 1,|c^{2}-10d^{2}|\neq 1$,所以要么它们都为2或者都为-2. \par 由于任何整数的平方模10为0,1,4,5,6,9,所以不会是2.所以$(a^{2}-10b^{2})mod(10)=(a^{2})mod(10)\neq 2$。所以假设错误。\par
3和$4\pm \sqrt{10}$的证明类似\par
~\\
37 令$a_{n}=2^{-n}ln(e_{n}-\frac{1}{2})=2^{-n}ln((e_{n-1}-\frac{1}{2})^{2}+\frac{1}{4})>2^{-n}ln((e_{n-1}-\frac{1}{2})^{2})=2^{-(n-1)}ln(e_{n-1}-\frac{1}{2})=a_{n-1}$ \par
令$b_{n}=2^{-n}ln(e_{n}+\frac{1}{2})=2^{-n}ln(e_{n-1}^{2}-e_{n-1}+\frac{3}{2})<2^{-n}ln(e_{n-1}^{2}+e_{n-1}+\frac{1}{4})=2^{-n}ln((e_{n-1}+\frac{1}{2})^{2})=b_{n-1}$ \par
所以$a_{n-1}<a_{n}<b_{n}<b_{n-1}$ \par
另外$e_{n}=\left \lfloor E^{2^{n}}+\frac{1}{2} \right \rfloor\Leftrightarrow e_{n}\leq E^{2^{n}}+\frac{1}{2}<e_{n}+1\Leftrightarrow e_{n}-\frac{1}{2}\leq E^{2^{n}}<e_{n}+\frac{1}{2}\Leftrightarrow 2^{-n}ln(e_{n}-\frac{1}{2})\leq lnE<2^{-n}ln(e_{n}+\frac{1}{2})\Leftrightarrow a_{n}\leq lnE<b_{n}$ \par
所以$E=\lim_{n\rightarrow oo}e^{a_{n}}$ \par
~\\
38 令$r=(n)mod(m)$,那么$a^{n}-b^{n}=(a^{m}-b^{m})(a^{n-m}+a^{n-2m}b^{m}+...+a^{r}b^{n-m-r})+b^{m\left \lfloor \frac{n}{m} \right \rfloor}(a^{r}-b^{r})$ \par
所以$Gcd(a^{n}-b^{n},a^{m}-b^{m})=Gcd((a^{n}-b^{n})mod(a^{m}-b^{m}),a^{m}-b^{m})=Gcd(b^{m\left \lfloor \frac{n}{m} \right \rfloor}(a^{r}-b^{r}),a^{m}-b^{m})$ \par
因为$a\perp b\rightarrow b^{m}\perp (a^{m}-b^{m})\rightarrow b^{m\left \lfloor \frac{n}{m} \right \rfloor}\perp (a^{m}-b^{m})$ \par
所以$Gcd(b^{m\left \lfloor \frac{n}{m} \right \rfloor}(a^{r}-b^{r}),a^{m}-b^{m})=Gcd(a^{r}-b^{r},a^{m}-b^{m})$ \par
一直这样下去可以得到$Gcd(a^{n}-b^{n},a^{m}-b^{m})=a^{Gcd(n,m)}-b^{Gcd(n,m)}$ \par
~\\
39 假设相等,设$m$的序列为$S_{m}=\left \{ m,a_{1},a_{2},...,a_{t},S(m) \right \}$,$m^{'}$的序列为$S_{m^{'}}=\left \{ m^{'},b_{1},b_{2},...,b_{u},S(m) \right \}$,令$\left \{ m,a_{1},a_{2},...,a_{t},S(m) \right \}\cap \left \{ m^{'},b_{1},b_{2},...,b_{u},S(m) \right \}=\left \{ c_{1},c_{2},..,c_{k},S_{m} \right \}=U$ \par
所以$\frac{\prod_{x\in S_{m}}x\prod_{y\in S_{m^{'}}}y}{\prod_{x\in U}x^{2}}$也是完全平方数,所以这时候$S(m)$不是最小的 \par
~\\
40 这里的$p$是一个素数。 \par
令$f(n)=\prod_{1\leq k \leq n,(k)mod(p)\ne 0}k=\frac{n!}{p^{\left \lfloor \frac{n}{p} \right \rfloor}\left \lfloor \frac{n}{p} \right \rfloor!}$ \par
那么$\frac{n!}{p^{\xi _{p}(n!)}}=f(n)f(\left \lfloor \frac{n}{p} \right \rfloor)f(\left \lfloor \frac{n}{p^{2}} \right \rfloor)...$ \par
而$f(n)\equiv a_{0}!((p-1)!)^{\left \lfloor \frac{n}{p} \right \rfloor}\equiv a_{0}!(-1)^{\left \lfloor \frac{n}{p} \right \rfloor}(mod(p))$ \par
$f(\left \lfloor \frac{n}{p} \right \rfloor)\equiv a_{1}!(-1)^{\left \lfloor \frac{n}{p^{2}} \right \rfloor}(mod(p))$ \par
$f(\left \lfloor \frac{n}{p^{2}} \right \rfloor)\equiv a_{2}!(-1)^{\left \lfloor \frac{n}{p^{3}} \right \rfloor}(mod(p))$ \par
乘起来可以得到$\frac{n!}{p^{\xi _{p}(n!)}}\equiv (-1)^{\xi _{p}(n!)}a_{0}!a_{1}!..a_{m}!(mod(p))$ \par
~\\
41 (1)假设$n^{2}\equiv -1(mod(p))\rightarrow (n^{2})^{\frac{p-1}{2}}=n^{p-1}\equiv -1(mod(p))$ ,这是矛盾的 \par
(2)$n=(\frac{p-1}{2})!$ \par
比如$p=13$,那么$(1)mod(13)=(-12)mod(13),(2)mod(13)=(-11)mod(13),..,(6)mod(13)=(-7)mod(13),$ \par
所以$n\equiv \left ( (-1)^{\frac{p-1}{2}}\prod_{1\leq k \leq \frac{p-1}{2}}(p-k)=\frac{(p-1)!}{n} \right )(mod(p))\rightarrow n^{2}\equiv \left ((p-1)!=1  \right )(mod(p))$ \par
~\\
42 A:$\frac{mn^{'}+m^{'}n}{nn{'}}$是最简分数 \par
B: $n\perp n^{'} $ \par
A$\rightarrow$ B: 假设$n,n^{'}$不互质,设$a=Gcd(n,n^{'}) > 1$,那么$n=pa,n^{'}=qa$,所以$mn^{'}+m^{'}n$和$nn^{'}$一定有公约数$a$.假设失败。所以一定有$n\perp n^{'}$ \par
B$\rightarrow$ A:假设$a=Gcd(mn^{'}+m^{'}n,nn^{'})>1$,由于$n\perp n^{'} $,不妨设$n=pa,a\perp n^{'},a\perp m $.所以$a\perp mn^{'},a|m^{'}n$,所以$Gcd(mn^{'}+m^{'}n,a)=1$,假设失败。所以$A$成立

~\\
43 函数$\rho (n)$的递推公式在题目23中。\par
(1)$n$为奇数时,$\begin{bmatrix} 0 &-1 \\  1 &1  \end{bmatrix}$=$L^{-1}R$。\par
(2)$n$为偶数时,可以用数学归纳法证明$\begin{bmatrix}
0 &-1 \\ 
1 &2\rho (n)+1 
\end{bmatrix}$=$R^{-\rho (n)}L^{-1}RL^{\rho(n)}$.不停用$\rho(2k)=\rho(k)+1$展开,前面的式子等价于证明$\begin{bmatrix}
0 &-1 \\ 
1 &2k+1 
\end{bmatrix}=R^{-k}L^{-1}RL^{k}$.前面一项以及乘以这个矩阵后的样子是分别是$...L\underbrace{RR...R}_{\rho(n)}\rightarrow ...R\underbrace{LL...L}_{\rho(n)}$ \par
~\\
44 数字0.3155和0.3165的分数是$\frac{631}{2000},\frac{633}{2000}$。在Stern-Brocot中在这个区间中最简单的分数是$\frac{6}{19}$ \par
~\\
45 $x^{2}\equiv x(mod(10^{n}))\Leftrightarrow x(x-1)\equiv 0(mod(10^{n}))\Leftrightarrow x(x-1)\equiv 0(mod(2^{n})),x(x-1)\equiv 0(mod(5^{n}))\Leftrightarrow (x)mod(2^{n})=0,1,(x)mod(5^{n})=0,1$ \par
所以满足条件的$x$最多有四个,其中两个是$0,1$,另外两个的形式为$t,10^{n}+1-t$ \par
46 (1)假设$j^{'}j-k^{'}k=Gcd(j,k)$,那么有$n^{j^{'}j}=n^{k^{'}k}n^{Gcd(j,k)}$,所以如果$n^{j^{'}j}=pm+1,n^{k^{'}k}=qm+1\rightarrow n^{Gcd(j,k)}=rm+1$ \par
(2)假设$n=pq$并且$p$是$n$的最小素因子(如果$n$为素数那么$p=n$)。所以$2^{p-1}\equiv 1(mod(p))$。\par
如果$2^{n}\equiv 1(mod(n))\rightarrow 2^{n}\equiv 1(mod(p))$。(如果$2^{n}\not\equiv 1(mod(p))$,不妨设为$x$,那么$2^{n}=kp+x$.如果仍然有$2^{n}\equiv 1(mod(n)$,那么有$2^{n}=rn+1$,所以$(kp+x)-(rn+1)=(k-rq)p+(x-1)=0$,显然不成立)\par
所以根据上面一个小题的结论,$2^{Gcd(p-1,n)}\equiv 1(mod(p))$。而由于$p$是$n$的最小素因子,所以$Gcd(p-1,n)=1$。这会导致错误。所以$2^{n}\not\equiv 1(mod(n))$ \par
~\\
47 $n^{m-1}\equiv 1(mod(m))\rightarrow n\perp m\rightarrow n^{t}\perp m\rightarrow \left ((n^{t})mod(m)  \right )\perp m,1\leq t <m$  \par
假设:如果对于所有的$1\leq t <m$,$\left (n^{t}  \right )mod(m)$不是各不相同的,比如对$1\leq x < y < m$有$\left ((n^{x})mod(m)  \right )=\left ((n^{y})mod(m)  \right )$,那么$n^{y-x}\equiv 1(mod(m))$,其中$y-x<m-1$ \par
根据题目46第一小题的结论,$n^{y-x}\equiv 1,n^{m-1}\equiv 1\rightarrow n^{Gcd(y-x,m-1)}\equiv 1$,而$Gcd(y-x,m-1)<m-1$.\par
令$k=min(y-x,Gcd(y-x,m-1))$,那么$k$一定能整除$m-1$.所以存在一个素数$p$以及一个整数$q$满足$kq=\frac{m-1}{p}$,而$\left ( n^{\frac{m-1}{p}}=n^{kq} \right )\equiv 1$,而这与题目给出的$n^{\frac{m-1}{p}}\not\equiv 1$矛盾了。所以上面的假设错误。\par 
对于所有的$1\leq t <m$,$\left (n^{t}  \right )mod(m)$各不相同,并且都与$m$互质,所以$1,2,3,...,m-1$都与$m$互质,所以$m$是素数 \par
~\\
48 将每个数字与其逆元相乘,得到1.所以可以不管这些数字。那么只剩下那些逆元是自己的数字,所以就是计算$\prod_{1\leq n < m,(n^{2})mod(m)=1}n$.根据$n^{2}\equiv 1(mod(m))$的解可以得到,当$m=4,p^{k},2p^{k}(p>2,k\geq 1)$答案为-1,否则为1 \par
~\\

49 (1)首先考虑$m<n$,此时答案为$\Phi (N)=\left (\sum_{k=1}^{N}\varphi (k)  \right )-1$,$m>n$也是这个,$m=n$时只有$m=n=1$成立,所以答案为$R(N)=2\Phi (N)-1$ \par
(2)由公式4.62可以得到$R(N)=2\Phi (N)-1=-1+\sum_{d\geq 1}\mu (d)\left \lfloor \frac{N}{d} \right \rfloor\left \lfloor \frac{N}{d}+1 \right \rfloor=\sum_{d\geq 1}\mu (d)\left \lfloor \frac{N}{d} \right \rfloor^{2}+\left (\sum_{d\geq 1}\mu (d)\left \lfloor \frac{N}{d} \right \rfloor-1  \right )$ \par
所以现在只需要满足$\sum_{d\geq 1}\mu (d)\left \lfloor \frac{N}{d} \right \rfloor=1$即可 \par
在公式4.61中,令$f(x)=[x\geq 1]\rightarrow g(N)=\sum_{d\geq 1}[\frac{N}{d}\geq 1]=N\rightarrow \sum_{d\geq 1}\mu (d)\left \lfloor \frac{N}{d} \right \rfloor=\sum_{d\geq 1}\mu (d)g(\left \lfloor \frac{N}{d} \right \rfloor)=\sum _{d\geq 1}\mu (d)g(\frac{N}{d})=f(N)=1$ \par
~\\
50 (1) 设$f$是任意一个函数。$\prod_{0\leq k < m}f(k)=\prod_{d|m}\prod_{0\leq k < m}f(k)[d=Gcd(k,m)]=\prod_{d|m}\prod_{0\leq k < m}f(k)[\frac{k}{d}\perp \frac{m}{d}]=\prod_{d|m}\prod_{0\leq k < \frac{m}{d}}f(kd)[k\perp \frac{m}{d}]=\prod_{d|m}\prod_{0\leq k < d}f(k\frac{m}{d})[k\perp d]$  \par
所以$z^{m}-1=\prod_{0\leq k < m}(z-\omega ^{k})=\prod_{d|m}\prod_{0\leq k < d,k\perp d}(z-\omega ^{\frac{km}{d}})=\prod_{d|m}\Psi _{m}(z)$ \par
最后一步成立是因为$\omega^{\frac{km}{d}}=e^{\frac{2\pi i}{m}*\frac{km}{d}}=e^{\frac{2\pi i}{d}*k}$ \par
(2)如果令$g(m)=z^{m}-1,f(m)=\Psi _{m}(z)$,也就是已知$g_{m}=\prod_{d|m}f(d)$而证明$f_{m}=\prod_{d|m}g(d)^{\mu (\frac{m}{d})}$,两边都取对数,就变成了公式4.56 \par
~\\
\end{document}
