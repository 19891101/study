\documentclass[onecolumn]{article}
\usepackage{xeCJK}
\usepackage{amsmath}
\usepackage{listings}
\usepackage{xcolor}
\setlength{\parindent}{0pt}
\renewcommand{\baselinestretch}{1.0}
\lstset{
	frame=tb, % draw a frame at the top and bottom of the code block
	showstringspaces=false, % don't mark spaces in strings
	numbers=left, % display line numbers on the left
	commentstyle=\color{green}, % comment color
	keywordstyle=\color{blue}, % keyword color
	stringstyle=\color{red} % string color
}
\usepackage[a4paper,left=20mm,right=20mm,top=15mm,bottom=15mm]{geometry}  


\begin{document}
1、当$n=2$时,区间$[2,n-1]$为空,所以当$n=2$时不能证明2匹马颜色相同。\par
~\\
2、三根柱子ABC。假设$n$个盘子的答案为$f(n)$.最后一个盘子一定是A->C->B,所以整个过程分为5步:
(1) 将上面$n-1$个盘子从A->C->B,即$f(n-1)$; \par 
(2)将第$n$个盘子放到C上;  \par 
 (3)将B上的$n-1$个盘子通过C移动到A,即$f(n-1)$; \par 
 (4)将C上的第$n$个盘子移动到B; \par 
 (5)最后将A上的$n-1$个盘子移动到B,即$f(n-1)$。 \par 
 所以$f(n)=3f(n-1)+2, f(1)=2$,所以$f(n)=3^{n}-1$. \par 
~\\

3、三根柱子的证明是类似的。下面只证明第一根柱子。数学归纳法:
(1)当$n=1$时,很明显,第一个柱子上出现过$2^1=2$种一个盘子的排列。 \par 
(2)假设$[1,n-1]$时,都满足情况; \par 
(3)对于$n$ 个盘子的情况,在第二题的第一步开始到第一步结束过程中,第一根柱子上出现过$n-1$个盘子的所有排列,此时有第$n$个盘子;在第二题的第五步开始到结束,第一根柱子上仍然出现过$n-1$个盘子的所有排列,此时没有第$n$个盘子。所以所有$n$个盘子的排列都出现过。
\par ~\\

4、数学归纳法:\par
(1)$n=1$时显然有$g(1) \le 2^{1}-1=1$ \par 
(2)假设$[1,n-1]$个都满足 \par 
(3)对于$n$个盘子,假设它在第三根上,那么$g(n)=g(n-1)$;否则假设它在第二根柱子上,那么可以将其他的$n-1$个先移动到第一根柱子上,需要$g(n-1)$,然后将第$n$个盘子移动到第三根上,然后再把第一根柱子上的$n-1$个盘子移动到第三根上,需要$2^{n-1}-1$步,所以$g(n)=g(n-1)+1+2^{n-1}-1 \le 2^{n-1}-1+1+2^{n-1}-1=2^{n}-1$ \par
所以不存在这样的排列。
\par ~\\
\end{document}
