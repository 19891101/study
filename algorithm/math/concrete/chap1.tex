\documentclass[onecolumn]{article}
\usepackage{xeCJK}
\usepackage{amsmath}
\usepackage{listings}
\usepackage{xcolor}
\setlength{\parindent}{0pt}
\renewcommand{\baselinestretch}{1.0}
\lstset{
	frame=tb, % draw a frame at the top and bottom of the code block
	showstringspaces=false, % don't mark spaces in strings
	numbers=left, % display line numbers on the left
	commentstyle=\color{green}, % comment color
	keywordstyle=\color{blue}, % keyword color
	stringstyle=\color{red} % string color
}
\usepackage[a4paper,left=20mm,right=20mm,top=15mm,bottom=15mm]{geometry}  


\begin{document}
1、当$n=2$时,区间$[2,n-1]$为空,所以当$n=2$时不能证明2匹马颜色相同。\par
~\\
2、三根柱子ABC。假设$n$个盘子的答案为$f(n)$.最后一个盘子一定是A->C->B,所以整个过程分为5步:
(1) 将上面$n-1$个盘子从A->C->B,即$f(n-1)$; \par 
(2)将第$n$个盘子放到C上;  \par 
 (3)将B上的$n-1$个盘子通过C移动到A,即$f(n-1)$; \par 
 (4)将C上的第$n$个盘子移动到B; \par 
 (5)最后将A上的$n-1$个盘子移动到B,即$f(n-1)$。 \par 
 所以$f(n)=3f(n-1)+2, f(1)=2$,所以$f(n)=3^{n}-1$. \par 
~\\

3、三根柱子的证明是类似的。下面只证明第一根柱子。数学归纳法:
(1)当$n=1$时,很明显,第一个柱子上出现过$2^1=2$种一个盘子的排列。 \par 
(2)假设$[1,n-1]$时,都满足情况; \par 
(3)对于$n$ 个盘子的情况,在第二题的第一步开始到第一步结束过程中,第一根柱子上出现过$n-1$个盘子的所有排列,此时有第$n$个盘子;在第二题的第五步开始到结束,第一根柱子上仍然出现过$n-1$个盘子的所有排列,此时没有第$n$个盘子。所以所有$n$个盘子的排列都出现过。
\par ~\\

4、数学归纳法:\par
(1)$n=1$时显然有$g(1) \le 2^{1}-1=1$ \par 
(2)假设$[1,n-1]$个都满足 \par 
(3)对于$n$个盘子,假设它在第三根上,那么$g(n)=g(n-1)$;否则假设它在第二根柱子上,那么可以将其他的$n-1$个先移动到第一根柱子上,需要$g(n-1)$,然后将第$n$个盘子移动到第三根上,然后再把第一根柱子上的$n-1$个盘子移动到第三根上,需要$2^{n-1}-1$步,所以$g(n)=g(n-1)+1+2^{n-1}-1 \le 2^{n-1}-1+1+2^{n-1}-1=2^{n}-1$ \par
所以不存在这样的排列。
\par ~\\

5、不能。两个圆最多两个交点,所以第四个圆最多跟前面的三个圆有6个交点,每个交点增加一个区域,所以最多有14个区域。
\par ~\\
6、三条直线组成三角形,有一个封闭区域。后面第$i$条直线与前面$i-1$条有$i-1$个交点,增加$i-2$个区域,所以答案为$\sum_{i=3}^{n}(i-2)=\frac{(n-1)(n-2)}{2}$
\par ~\\

7、$H(1)=J(2)-J(1)=0 \ne 2$。所以用归纳法的话,初始条件不成立。
\par ~\\

8、$Q_{2}=\frac{1+\beta }{\alpha }$,$Q_{3}=\frac{1+\alpha +\beta }{\alpha \beta }$,$Q_{4}=\frac{1+\alpha }{\beta }$,$Q_{5}=\alpha $,$Q_{6}=\beta $。所以会构成一个循环。
\par ~\\

9、(1)将$x_{n}$带入,明显等式成立。\par
(2)由于$P(n)$成立,即$\prod_{i=1}^{n}x_{i} \leq (\frac{\sum_{i=1}^{n}x_{i}}{n})^{n}$,所以有$\prod_{i=n+1}^{2n}x_{i}\leq (\frac{\sum_{i=n+1}^{2n}x_{i}}{n})^{n}$。将两个式子相乘得到: 
$\prod_{i=1}^{2n}x_{i}\leq (\frac{\sum_{i=1}^{2n}x_{i}}{n}\frac{\sum_{i=n+1}^{2n}x_{i}}{n})^{n}$
而$xy\leq (\frac{x+y}{2})^2$,所以$ \frac{\sum_{i=1}^{2n}x_{i}}{n}\frac{\sum_{i=n+1}^{2n}x_{i}}{n}\leq \left (\frac{\sum_{i=1}^{2n}x_{i}}{2n}  \right )^{2}$, 从而得到$P(2n)$成立\par
(3)$P(2)$->$P(4)$->$P(3)$->$P(6)$->...(这是基于第一个条件) \par
~\\
下面假设没有条件1的特殊情况,关于这个不等式的证明: \par
\begin{itemize}
	\item 设$f(x)=e^{x-1}-x$,通过求导可以看出$f(x)$在$x=1$时取得最小值0,所以$f(x)\geq 0$,即$x \leq e^{x-1}$。
	\item 对于要证明的式子,如果存在某个$x_{i}=0$,那么显然成立.下面假设都大于0
	\item 令$a=\frac{\sum_{i=1}^{n}x_{i}}{n}>0$,那么对任意的$x_{i}$有$\frac{x_{i}}{a}\leq e^{\frac{x_{i}}{a}-1}$ 
	\item 所有式子相乘得到:$\frac{\prod_{i=1}^{n}x_{i}}{a^n}\leq e^{\sum_{i=1}^{n}\frac{x_{i}}{a}-n}=e^{n-n}=1$,所以$\prod_{i=1}^{n}x_{i}\leq a^{n}=\left ( \frac{\sum_{i=1}^{n}x_{i}}{n} \right )^n$
\end{itemize}

\par ~\\
10、对于一个环A->B->C->A,$Q_{n}$表示A->B或者B->C或者C->A,而$R_{n}$表示A->B->C或者B->C->A或者C->A->B。\par
(1)对于$Q_{n}$来说,分三步:(1)将上面$n-1$个从A->B->C;(2)将第$n$个从A到B,(3)将剩下的$n-1$个从C->A->B。所以$Q_{n}=2R_{n-1}+1$ \par
(2)对于$R_{n}$来说,分五步:(1)将上面$n-1$个从B->C->A;(2)将第$n$个从B到C,(3)将A上的$n-1$个从A->B,(4)将C上的第$n$个放到A;(5)将B上的$n-1$个从B->C->A.所以$R_{n}=R_{n-1}+1+Q_{n-1}+1+R_{n-1}=Q_{n}+Q_{n-1}+1$ 
\par ~\\
11、(1)令$P_{n}$表示$2n$个圆盘从A移动到C的解.分为三步:(1)将上面的$2(n-1)$个从A挪到B,需要$P_{n-1}$;(2)将A上剩下的两个移动到C,需要2步;(3)将B上的盘子移动到C,需要$P_{n-1}$。所以$P_{n}=2P_{n-1}+2,P_{1}=2$,所以$P_{n}=2^{n+1}-2$ \par
(2)设$Q_{n}$表示$2n$个盘子的答案,分为7步:(1)将上面$2(n-1)$个盘子移动到C,需要$P_{n-1}$;(2)第大小为$n$的上面一个盘子移动到B;(3)将C上的$2(n-1)$个盘子移动到B,需要$P_{n-1}$;(4)将最后一个盘子移动到C;(5)将B的上面$2(n-1)$个盘子移动到A,需要$P_{n-1}$;(6)将B上的剩下的一个盘子移动到C;(7)将A的$2(n-1)$个盘子移动到C,需要$P_{n-1}$,所以$Q_{n}=4P_{n-1}+3=2^{n+2}-5$.上面的$2(n-1)$个盘子一共挪动了四次。每一次两个相同的会反一下,所以四次会跟开始的时候的顺序一样。所以所有的盘子跟开始的顺序都是一样的。
\par ~\\
12、 $A(m_{1},m_{2},..,m_{n})=2A(m_{1},..,m_{n-1})+m_{n}=\sum_{i=1}^{n}2^{n-i}m_{i}$
\par ~\\
13、令$F(n)$表示$n$个$Z$型线的答案。$L(n)$表示$n$条直线的答案。$L(n)=\frac{n(n+1)}{2}+1$.首先将一个$Z$型线看作三条直线.但是一个$Z$型线比三条互不平行的直线少了5个区域,所以$F(n)=L(3n)-5n=\frac{9n^{2}-7n}{2}+1$.少的5个区域包括:
\begin{itemize}
	\item 两条平行的直线之间少了1个,由4个变为3个
	\item 中间的那条跟每个平行的线之间少了2个,由4个变为2个,共少了2*2=4个
\end{itemize}
\par ~\\
14、$L(n)$表示$n$条直线在二维平面划分的区域的个数。$L(n)=\frac{n(n+1)}{2}+1$.那么在三维空间中,第$n$次切割所增加的块的个数就是这一次切割处的面上的区域数,所以$P(n)=P(n-1)+L(n-1),P(1)=2$,所以$P(n)=\frac{1}{6}(n+1)(n^{2}-n+6)$
\par ~\\
15、首先$I(2)=2,I(3)=1$.对于$n>=4$来说,\par
\begin{itemize}
	\item 如果$n$是偶数,那么第一轮将删掉2,4,6,...,n。下面重新从1开始计数,所以此时$I(n)=2I(\frac{n}{2})-1$
	\item 如果$n$是奇数,那么第一轮可以删掉 2,4,6,...,$n-1$,1,接下来从3开始计数,所以此时$I(n)=2I(\frac{n-1}{2})+1$
\end{itemize}
也可以这样写$I(2)=2,I(3)=1,I(2n)=2I(n)-1,I(2n+1)=2I(n)+1, n\geq 2$
\par ~\\
\end{document}
