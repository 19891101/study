\documentclass[onecolumn]{article}
\usepackage{xeCJK}
\usepackage{amsmath}
\usepackage{listings}
\usepackage{xcolor}
\setlength{\parindent}{0pt}
\renewcommand{\baselinestretch}{1.0}
\lstset{
	frame=tb, % draw a frame at the top and bottom of the code block
	showstringspaces=false, % don't mark spaces in strings
	numbers=left, % display line numbers on the left
	commentstyle=\color{green}, % comment color
	keywordstyle=\color{blue}, % keyword color
	stringstyle=\color{red} % string color
}
\usepackage[a4paper,left=20mm,right=20mm,top=15mm,bottom=15mm]{geometry}  


\begin{document}
1、下面的是下界,上面的是上界,所以这个取值范围为空,答案应该是0 \par
~\\
2、$|x|$ \par
~\\
3、$\sum_{0\leq k\leq 5}a_{k}=a_{0}+a_{1}+a_{2}+a_{3}+a_{4}+a_{5}$ \par
$\sum_{0\leq k^{2}\leq 5}a_{k^{2}}=\sum_{k=-2}^{2}a_{k^{2}}=a_{4}+a_{1}+a_{0}+a_{1}+a_{4}$ \par
~\\
4、$\sum_{1\leq i<j<k\leq 4}a_{ijk}=\sum_{i=1}^{2}\sum_{j=i+1}^{3}\sum_{k=j+1}^{4}a_{ijk}=((a_{123}+a_{124})+a_{134})+a_{234}$\par

$\sum_{1\leq i<j<k\leq 4}a_{ijk}=\sum_{k=3}^{4}\sum_{j=2}^{k-1}\sum_{i=1}^{j-1}a_{ijk}=a_{123}+(a_{124}+(a_{134}+a_{234}))$\par
~\\
5、两个求和符号用了同样的下标符号,其实它们是不同的,所以不能约分。\par
~\\
6、$[1\leq j\leq n](n-j+1)$\par
~\\
7、$mx^{\overline{m-1}}$\par
~\\
8、当$m>0$时,为0;当$m=0$为1;当$m<0$时为$\frac{1}{|m|!}$\par
~\\
9、$x^{\overline{m+n}}=x^{\overline{m}}(x+m)^{\overline{n}}$。令$m=-n$可以得到:$x^{\overline{-n}}=\frac{1}{(x-n)^{\overline{n}}}=\frac{1}{(x-1)^{\underline{b}}}$ \par
~\\
10、$u\Delta v+E_{v}\Delta u=v\Delta u+E_{u}\Delta v$,这样就对称了。\par
~\\
11、$a_{n}b_{n}-a_{0}b_{0}-\sum_{0\leq k < n}a_{k+1}(b_{k+1}-b_{k})$\par
$=a_{n}b_{n}-a_{0}b_{0}-\sum_{0\leq k < n}a_{k+1}b_{k+1}+\sum_{0\leq k < n}a_{k+1}b_{k}$\par
$=a_{n}b_{n}-a_{0}b_{0}-\sum_{1\leq k \leq n}a_{k}b_{k}+\sum_{0\leq k < n}a_{k+1}b_{k}$\par
$=-\sum_{0\leq k < n}a_{k}b_{k}+\sum_{0\leq k < n}a_{k+1}b_{k}$\par
$=\sum_{0\leq k < n}(a_{k+1}-a_{k})b_{k}$\par
~\\
12、从两点证明:\par
\begin{itemize}
	\item 对于两个不同的$k_{1},k_{2}$,$p(k_{1})\ne p(k_{2})$
	\item 对于一个整数$n$,一定存在一个整数$k$,满足$p(k)=n$.如果$k$是奇数,那么$p(2t+1)=2t+1-c$,和跟$c$的奇偶性相反;如果$k$是偶数,那么$p(2t)=2t-c$,和跟$c$的奇偶性相同。这两个里面一定会存在一个等于$n$
\end{itemize}
~\\
13、令$R_{0}=\alpha,R_{n}=R_{n-1}+(-1)^{n}(\beta+\gamma n+\delta n^{2})$,所以$R_{n}=A(n)\alpha+B(n)\beta+C_{n}\gamma +D_{n}\delta $ \par
(1)令$R_{n}=1$可以得到:$\alpha=1,\beta=\gamma =\delta =0$,所以$A_{n}=1$ \par
(2)令$R_{n}=(-1)^{n}$,可以得到:$\alpha=1,\beta=2,\gamma=\delta=0$,所以$A(n)+2B(n)=(-1)^{n}$ \par
(3)令$R_{n}=(-1)^{n}n$,可以得到:$-B(n)+2C(n)=(-1)^{n}n$ \par
(4)令$R_{n}=(-1)^{n}n^{2}$,可以得到:$B(n)-2C(n)+2D(n)=(-1)^{n}n^{2}$. \par
令$\alpha=\beta=\gamma=0,\delta=1,R_{n}=(-1)^{n}n^{2}$,那么有:\par
\begin{itemize}
	\item $R_{n}=D(n)$
	\item $R_{0}=0$
	\item $R_{1}=(-1)^{1}1^{2}$
	\item $R_{2}=R_{1}+(-1)^{2}2^{2}=(-1)^{1}1^{2}+(-1)^{2}2^{2}$
	\item 所以$D(n)=R_{n}=\sum_{k=0}^{n}(-1)^{k}k^{2}$
\end{itemize}
因此$\sum_{k=0}^{n}(-1)^{k}k^{2}=D(n)=\frac{(-1)^{n}n^{2}-B(n)+2C(n)}{2}$\par
$=\frac{(-1)^{n}n^{2}+(-1)^{n}n}{2}=\frac{(-1)^{n}(n^{2}+n)}{2}$\par
~\\
14、$\sum_{1\leq j \leq k \leq n}2^{k}$\par
$=\sum_{1\leq j \leq n}\sum_{j\leq k \leq n}2^{k}$\par
$=\sum_{1\leq j \leq n}(2^{n+1}-2^{j})$\par
$=n2^{n+1}-\sum_{1\leq j \leq n}2^{j}$\par
$=n2^{n+1}-(2^{n+1}-2)$\par
$=(n-1)2^{n+1}+2$\par
~\\
15、$\sum_{k=1}^{n}k^{3}+\sum_{k=1}^{n}k^{2}$ \par
$=\sum_{k=1}^{n}(k^{3}+k^{2})$\par
$=\sum_{k=1}^{n}k*k(k+1)$\par
$=\sum_{k=1}^{n}k\sum_{j=1}^{k}2j$\par
$=2\sum_{1\leq j \leq k \leq n}jk$\par
$=\sum_{1\leq j,k \leq n}jk+\sum_{1\leq j=k \leq n}jk=\left (\sum_{1\leq k \leq n}k  \right )^{2}+\sum_{k=1}^{n}k^{2}$\par
$=(\frac{n(n+1)}{2})^{2}+\sum_{k=1}^{n}k^{2}$\par
所以$\sum_{k=1}^{n}k^{3}=(\frac{n(n+1)}{2})^{2}$\par
~\\
16、$x^{\underline{n}}(x-n)^{\underline{m}}=x^{\underline{m}}(x-m)^{\underline{n}}=x^{\underline{n+m}}$ \par
~\\
17、两个式子类似,只证明第一个。首先给出一些总结:\par
\begin{itemize}
	\item 当$m>0$时,有$x^{\overline{m}}=x(x+1)(x+2)..(x+m-2)(x+m-1)$
	\item 当$m=0$时,有$x^{\overline{0}}=1$
	\item 当$m<0$时,有$x^{\overline{m}}=\frac{1}{(x-1)(x-2)...(x-(|m|-1))(x-|m|)}$
	\item 当$m>$时,有$x^{\underline{m}}=x(x-1)(x-2)..(x-(m-2))(x-(m-1))$
	\item 当$m=0$时,有$x^{\underline{0}}=1$
	\item 当$m<0$时,有$x^{\underline{m}}=\frac{1}{(x+1)(x+2)...(x+(|m|-1))(x+|m|)}$
\end{itemize}
 (1)$m=0$时显然都是1 \par
 (2)$m>0$时,
\begin{itemize}
	\item $(-1)^{m}(-x)^{\underline{m}}=(-1)^{m}(-x)(-x-1)(-x-2)...(-x-(m-1))=x(x+1)(x+2)...(x+m-1)=x^{\overline{m}}$
	\item $(x+m-1)^{\underline{m}}=(x+m-1)(x+m-2)...(x+1)x=x^{\overline{m}}$
	\item $\frac{1}{(x-1)^{\underline{-m}}}=(x-1+1)(x-1+2)...(x-1+m)=x^{\overline{m}}$
\end{itemize}
(3)当$m<0$时,不妨令$m=-m$,
\begin{itemize}
	\item $(-1)^{-m}(-x)^{\underline{-m}}=\frac{1}{(-1)^{m}}*\frac{1}{(-x+1)(-x+2)...(-x+m)}=\frac{1}{(x-1)(x-2)(x-3)..(x-m)}=x^{\overline{-m}}$
	\item $(x-m-1)^{\underline{-m}}=\frac{1}{(x-m-1+1)(x-m-1+2)...(x-m-1+m)}=x^{\overline{-m}}$
	\item $\frac{1}{(x-1)^{\underline{m}}}=\frac{1}{(x-1)(x-1-1)...(x-1-(m-1))}=x^{\overline{-m}}$
\end{itemize}
~\\
18、
\begin{itemize}
	\item $p$: $\sum_{k \in K}a_{k}$绝对收敛
	\item $q$: 存在有界常数$B$使得任意有限子集$F \in K$有$\sum_{k \in F}|a_{k}| \leq B$
\end{itemize}

(1) $p\rightarrow q$:\par 
若$\sum_{k \in K}a_{k}$绝对收敛,那么有$\sum_{k\in K}\Re a_{k},\sum_{k\in K}\Im  a_{k}$分别绝对收敛,而$|a_{k}|\leq (\Re a_{k})^{+}+(\Re a_{k})^{-}+(\Im  a_{k})^{+}+(\Im  a_{k})^{-}$,所以$\sum_{k\in F}|a_{k}|\leq \sum_{k\in F}((\Re a_{k})^{+}+(\Re a_{k})^{-}+(\Im  a_{k})^{+}+(\Im  a_{k})^{-})$,而后者绝对收敛,所以存在有界常数$B$满足条件;\par
~\\
(2) $q\rightarrow p$:\par
由于$(\Re a_{k})^{+}\leq |a_{k}|,(\Re a_{k})^{-}\leq |a_{k}|,(\Im  a_{k})^{+}\leq |a_{k}|,(\Im  a_{k})^{-}\leq |a_{k}|$,所以对于任意的$F$存在有界常数$X,Y,Z,W$使得$\sum_{k\in F}(\Re a_{k})^{+}\leq X,\sum_{k\in F}(\Re a_{k})^{-}\leq Y,\sum_{k\in F}(\Im  a_{k})^{+}\leq Z,\sum_{k\in F}(\Im  a_{k})^{-}\leq W$,所以$\sum_{k\in K}\Re a_{k},\sum_{k\in K}\Im a_{k}$都是绝对收敛的,所以$\sum_{k \in K}a_{k}$绝对收敛 \par
~\\
\end{document}
