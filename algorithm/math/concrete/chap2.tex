\documentclass[onecolumn]{article}
\usepackage{xeCJK}
\usepackage{amsmath}
\usepackage{listings}
\usepackage{xcolor}
\setlength{\parindent}{0pt}
\renewcommand{\baselinestretch}{1.0}
\lstset{
	frame=tb, % draw a frame at the top and bottom of the code block
	showstringspaces=false, % don't mark spaces in strings
	numbers=left, % display line numbers on the left
	commentstyle=\color{green}, % comment color
	keywordstyle=\color{blue}, % keyword color
	stringstyle=\color{red} % string color
}
\usepackage[a4paper,left=20mm,right=20mm,top=15mm,bottom=15mm]{geometry}  


\begin{document}
1、下面的是下界,上面的是上界,所以这个取值范围为空,答案应该是0 \par
~\\
2、$|x|$ \par
~\\
3、$\sum_{0\leq k\leq 5}a_{k}=a_{0}+a_{1}+a_{2}+a_{3}+a_{4}+a_{5}$ \par
$\sum_{0\leq k^{2}\leq 5}a_{k^{2}}=\sum_{k=-2}^{2}a_{k^{2}}=a_{4}+a_{1}+a_{0}+a_{1}+a_{4}$ \par
~\\
4、$\sum_{1\leq i<j<k\leq 4}a_{ijk}=\sum_{i=1}^{2}\sum_{j=i+1}^{3}\sum_{k=j+1}^{4}a_{ijk}=((a_{123}+a_{124})+a_{134})+a_{234}$\par

$\sum_{1\leq i<j<k\leq 4}a_{ijk}=\sum_{k=3}^{4}\sum_{j=2}^{k-1}\sum_{i=1}^{j-1}a_{ijk}=a_{123}+(a_{124}+(a_{134}+a_{234}))$\par
~\\
5、两个求和符号用了同样的下标符号,其实它们是不同的,所以不能约分。\par
~\\
6、$[1\leq j\leq n](n-j+1)$\par
~\\
7、$mx^{\overline{m-1}}$\par
~\\
8、当$m>0$时,为0;当$m=0$为1;当$m<0$时为$\frac{1}{|m|!}$\par
~\\
9、$x^{\overline{m+n}}=x^{\overline{m}}(x+m)^{\overline{n}}$。令$m=-n$可以得到:$x^{\overline{-n}}=\frac{1}{(x-n)^{\overline{n}}}=\frac{1}{(x-1)^{\underline{b}}}$ \par
~\\
10、$u\Delta v+E_{v}\Delta u=v\Delta u+E_{u}\Delta v$,这样就对称了。\par
~\\
11、$a_{n}b_{n}-a_{0}b_{0}-\sum_{0\leq k < n}a_{k+1}(b_{k+1}-b_{k})$\par
$=a_{n}b_{n}-a_{0}b_{0}-\sum_{0\leq k < n}a_{k+1}b_{k+1}+\sum_{0\leq k < n}a_{k+1}b_{k}$\par
$=a_{n}b_{n}-a_{0}b_{0}-\sum_{1\leq k \leq n}a_{k}b_{k}+\sum_{0\leq k < n}a_{k+1}b_{k}$\par
$=-\sum_{0\leq k < n}a_{k}b_{k}+\sum_{0\leq k < n}a_{k+1}b_{k}$\par
$=\sum_{0\leq k < n}(a_{k+1}-a_{k})b_{k}$\par
~\\
12、从两点证明:\par
\begin{itemize}
	\item 对于两个不同的$k_{1},k_{2}$,$p(k_{1})\ne p(k_{2})$
	\item 对于一个整数$n$,一定存在一个整数$k$,满足$p(k)=n$.如果$k$是奇数,那么$p(2t+1)=2t+1-c$,和跟$c$的奇偶性相反;如果$k$是偶数,那么$p(2t)=2t-c$,和跟$c$的奇偶性相同。这两个里面一定会存在一个等于$n$
\end{itemize}
~\\
13、令$R_{0}=\alpha,R_{n}=R_{n-1}+(-1)^{n}(\beta+\gamma n+\delta n^{2})$,所以$R_{n}=A(n)\alpha+B(n)\beta+C_{n}\gamma +D_{n}\delta $ \par
(1)令$R_{n}=1$可以得到:$\alpha=1,\beta=\gamma =\delta =0$,所以$A_{n}=1$ \par
(2)令$R_{n}=(-1)^{n}$,可以得到:$\alpha=1,\beta=2,\gamma=\delta=0$,所以$A(n)+2B(n)=(-1)^{n}$ \par
(3)令$R_{n}=(-1)^{n}n$,可以得到:$-B(n)+2C(n)=(-1)^{n}n$ \par
(4)令$R_{n}=(-1)^{n}n^{2}$,可以得到:$B(n)-2C(n)+2D(n)=(-1)^{n}n^{2}$. \par
令$\alpha=\beta=\gamma=0,\delta=1,R_{n}=(-1)^{n}n^{2}$,那么有:\par
\begin{itemize}
	\item $R_{n}=D(n)$
	\item $R_{0}=0$
	\item $R_{1}=(-1)^{1}1^{2}$
	\item $R_{2}=R_{1}+(-1)^{2}2^{2}=(-1)^{1}1^{2}+(-1)^{2}2^{2}$
	\item 所以$D(n)=R_{n}=\sum_{k=0}^{n}(-1)^{k}k^{2}$
\end{itemize}
因此$\sum_{k=0}^{n}(-1)^{k}k^{2}=D(n)=\frac{(-1)^{n}n^{2}-B(n)+2C(n)}{2}$\par
$=\frac{(-1)^{n}n^{2}+(-1)^{n}n}{2}=\frac{(-1)^{n}(n^{2}+n)}{2}$\par
~\\
14、$\sum_{1\leq j \leq k \leq n}2^{k}$\par
$=\sum_{1\leq j \leq n}\sum_{j\leq k \leq n}2^{k}$\par
$=\sum_{1\leq j \leq n}(2^{n+1}-2^{j})$\par
$=n2^{n+1}-\sum_{1\leq j \leq n}2^{j}$\par
$=n2^{n+1}-(2^{n+1}-2)$\par
$=(n-1)2^{n+1}+2$\par
~\\
15、$\sum_{k=1}^{n}k^{3}+\sum_{k=1}^{n}k^{2}$ \par
$=\sum_{k=1}^{n}(k^{3}+k^{2})$\par
$=\sum_{k=1}^{n}k*k(k+1)$\par
$=\sum_{k=1}^{n}k\sum_{j=1}^{k}2j$\par
$=2\sum_{1\leq j \leq k \leq n}jk$\par
$=\sum_{1\leq j,k \leq n}jk+\sum_{1\leq j=k \leq n}jk=\left (\sum_{1\leq k \leq n}k  \right )^{2}+\sum_{k=1}^{n}k^{2}$\par
$=(\frac{n(n+1)}{2})^{2}+\sum_{k=1}^{n}k^{2}$\par
所以$\sum_{k=1}^{n}k^{3}=(\frac{n(n+1)}{2})^{2}$\par
~\\
16、$x^{\underline{n}}(x-n)^{\underline{m}}=x^{\underline{m}}(x-m)^{\underline{n}}=x^{\underline{n+m}}$ \par
~\\
17、两个式子类似,只证明第一个。首先给出一些总结:\par
\begin{itemize}
	\item 当$m>0$时,有$x^{\overline{m}}=x(x+1)(x+2)..(x+m-2)(x+m-1)$
	\item 当$m=0$时,有$x^{\overline{0}}=1$
	\item 当$m<0$时,有$x^{\overline{m}}=\frac{1}{(x-1)(x-2)...(x-(|m|-1))(x-|m|)}$
	\item 当$m>$时,有$x^{\underline{m}}=x(x-1)(x-2)..(x-(m-2))(x-(m-1))$
	\item 当$m=0$时,有$x^{\underline{0}}=1$
	\item 当$m<0$时,有$x^{\underline{m}}=\frac{1}{(x+1)(x+2)...(x+(|m|-1))(x+|m|)}$
\end{itemize}
 (1)$m=0$时显然都是1 \par
 (2)$m>0$时,
\begin{itemize}
	\item $(-1)^{m}(-x)^{\underline{m}}=(-1)^{m}(-x)(-x-1)(-x-2)...(-x-(m-1))=x(x+1)(x+2)...(x+m-1)=x^{\overline{m}}$
	\item $(x+m-1)^{\underline{m}}=(x+m-1)(x+m-2)...(x+1)x=x^{\overline{m}}$
	\item $\frac{1}{(x-1)^{\underline{-m}}}=(x-1+1)(x-1+2)...(x-1+m)=x^{\overline{m}}$
\end{itemize}
(3)当$m<0$时,不妨令$m=-m$,
\begin{itemize}
	\item $(-1)^{-m}(-x)^{\underline{-m}}=\frac{1}{(-1)^{m}}*\frac{1}{(-x+1)(-x+2)...(-x+m)}=\frac{1}{(x-1)(x-2)(x-3)..(x-m)}=x^{\overline{-m}}$
	\item $(x-m-1)^{\underline{-m}}=\frac{1}{(x-m-1+1)(x-m-1+2)...(x-m-1+m)}=x^{\overline{-m}}$
	\item $\frac{1}{(x-1)^{\underline{m}}}=\frac{1}{(x-1)(x-1-1)...(x-1-(m-1))}=x^{\overline{-m}}$
\end{itemize}
~\\
18、\begin{itemize}
	\item $p$: $\sum_{k \in K}a_{k}$绝对收敛
	\item $q$: 存在有界常数$B$使得任意有限子集$F \in K$有$\sum_{k \in F}|a_{k}| \leq B$
\end{itemize}
(1) $p\rightarrow q$:\par 
若$\sum_{k \in K}a_{k}$绝对收敛,那么有$\sum_{k\in K}\Re a_{k},\sum_{k\in K}\Im  a_{k}$分别绝对收敛,而$|a_{k}|\leq (\Re a_{k})^{+}+(\Re a_{k})^{-}+(\Im  a_{k})^{+}+(\Im  a_{k})^{-}$,所以$\sum_{k\in F}|a_{k}|\leq \sum_{k\in F}((\Re a_{k})^{+}+(\Re a_{k})^{-}+(\Im  a_{k})^{+}+(\Im  a_{k})^{-})$,而后者绝对收敛,所以存在有界常数$B$满足条件;\par
(2) $q\rightarrow p$:\par
由于$(\Re a_{k})^{+}\leq |a_{k}|,(\Re a_{k})^{-}\leq |a_{k}|,(\Im  a_{k})^{+}\leq |a_{k}|,(\Im  a_{k})^{-}\leq |a_{k}|$,所以对于任意的$F$存在有界常数$X,Y,Z,W$使得$\sum_{k\in F}(\Re a_{k})^{+}\leq X,\sum_{k\in F}(\Re a_{k})^{-}\leq Y,\sum_{k\in F}(\Im  a_{k})^{+}\leq Z,\sum_{k\in F}(\Im  a_{k})^{-}\leq W$,所以$\sum_{k\in K}\Re a_{k},\sum_{k\in K}\Im a_{k}$都是绝对收敛的,所以$\sum_{k \in K}a_{k}$绝对收敛 \par
19、\begin{itemize}
	\item $a_{n}=2,b_{n}=n$,所以$s_{n}=\frac{a_{1}a_{2}...a_{n-1}}{b_{2}b_{3}...b_{n}}=\frac{2^{n-1}}{n!}$
	\item 两边同时乘以$s_{n}$得到:$\frac{2^{n}}{n!}T_{n}=\frac{2^{n-1}}{(n-1)!}T_{n-1}+3*2^{n-1}$,$T_{0}=5\rightarrow T_{1}=4$
	\item 令$P_{n}=\frac{2^{n}}{n!}T_{n}$,那么有$P_{n}=P_{n-1}+3*2^{n-1},P_{1}=8$,所以$P_{n}=3*(2^{n-1}+2^{n-2}+...+2^{1})+8=3*2^{n}+2$
	\item 所以$T_{n}=(3*2^{n}+2)*\frac{n!}{2^{n}}$,验证可得对于$n=0$也满足条件。
\end{itemize}
~\\
20、$\sum_{k=0}^{n}kH_{k}+(n+1)H_{n+1}$ \par
$=\sum_{k=0}^{n}(k+1)H_{k+1}$\par
$=\sum_{k=0}^{n}kH_{k+1}+\sum_{k=0}^{n}H_{k+1}$\par
$=\sum_{k=0}^{n}k(H_{k}+\frac{1}{k+1})+\sum_{k=0}^{n}(H_{k}+\frac{1}{k+1})$\par
$=\sum_{k=0}^{n}kH_{k}+\sum_{k=0}^{n}\frac{k}{k+1}+\sum_{k=0}^{n}H_{k}+\sum_{k=0}^{n}\frac{1}{k+1}$\par
$=\sum_{k=0}^{n}kH_{k}+\sum_{k=0}^{n}H_{k}+n+1$\par
所以$\sum_{k=0}^{n}H_{k}=(n+1)(H_{n+1}-1)$\par
~\\
21、 (1)$S_{n}$的计算: \par
一方面,$S_{n+1}=\sum_{k=0}^{n+1}(-1)^{n+1-k}=1+\sum_{k=0}^{n}(-1)^{n+1-k}=1-\sum_{k=0}^{n}(-1)^{n-k}=1-S_{n}$\par
另一方面,$S_{n+1}=\sum_{k=0}^{n+1}(-1)^{n+1-k}=(-1)^{n+1}+\sum_{k=1}^{n+1}(-1)^{n+1-k}=(-1)^{n+1}+\sum_{k=0}^{n}(-1)^{n-k}=(-1)^{n+1}+S_{n}$ \par
所以$S_{n}=\frac{1+(-1)^{n}}{2}$ \par
~\\
(2)$T_{n}$的计算: \par 一方面$T_{n+1}=\sum_{k=0}^{n+1}(-1)^{n+1-k}k=\sum_{k=1}^{n+1}(-1)^{n+1-k}k=\sum_{1\leq k+1\leq n+1}(-1)^{n+1-(k+1)}(k+1)$\par
$=\sum_{k=0}^{n}(-1)^{n-k}(k+1)=\sum_{k=0}^{n}(-1)^{n-k}k+\sum_{k=0}^{n}(-1)^{n-k}=T_{n}+S_{n}$ \par
另一方面,$T_{n+1}=\sum_{k=0}^{n+1}(-1)^{n+1-k}k=n+1+\sum_{k=0}^{n}(-1)^{n+1-k}k=n+1-\sum_{k=0}^{n}(-1)^{n-k}k=n+1-T_{n}$ \par
所以$T_{n}=\frac{n+1-S_{n}}{2}=\frac{2n+1-(-1)^n}{4}$ \par
~\\
(3)$U_{n}$的计算:\par
一方面$U_{n+1}=\sum_{k=0}^{n+1}(-1)^{n+1-k}k^{2}=\sum_{k=1}^{n+1}(-1)^{n+1-k}k^{2}=\sum_{k=0}^{n}(-1)^{n-k}(k+1)^{2}$\par
$=\sum_{k=0}^{n}(-1)^{n-k}k^{2}+2T_{n}+S_{n}=U_{n}+n+1$\par
另一方面,$U_{n+1}=\sum_{k=0}^{n+1}(-1)^{n+1-k}k^{2}=(n+1)^{2}+\sum_{k=0}^{n}(-1)^{n+1-k}k^{2}=(n+1)^{2}-U_{n}$ \par
所以$U_{n}=\frac{n(n+1)}{2}$ \par
~\\
22、直接证明下面的一般式: \par
$\sum_{1\leq j < k \leq n}(a_{j}b_{k}-a_{k}b_{j})(A_{j}B_{k}-A_{k}B_{j})$ \par
$=\frac{1}{2}\sum_{1\leq j,k \leq n}(a_{j}b_{k}-a_{k}b_{j})(A_{j}B_{k}-A_{k}B_{j})$ \par
$=\frac{1}{2}\sum_{1\leq j,k \leq  n}(a_{j}b_{k}A_{j}B_{k}-a_{k}b_{j}A_{j}B_{k}-a_{j}b_{k}A_{k}B_{j}+a_{k}b_{j}A_{k}B_{j})$ \par
$=\sum_{1\leq j,k \leq n}(a_{j}b_{k}A_{j}B_{k}-a_{k}b_{j}A_{j}B_{k})$\par
$=\left (\sum_{k=1}^{n}a_{k}A_{k}  \right )\left (\sum_{k=1}^{n}b_{k}B_{k}  \right )-\left (\sum_{k=1}^{n}a_{k}B_{k}  \right )\left (\sum_{k=1}^{n}b_{k}A_{k}  \right )$\par
当$a_{i}=A_{i},b_{i}=B_{i}$时可得到上面的等式 \par
~\\
23、(1)$\sum_{k=1}^{n}\frac{2k+1}{k(k+1)}=\sum_{k=1}^{n}\frac{k+(k+1)}{k(k+1)}=\sum_{k=1}^{n}(\frac{1}{k}+\frac{1}{k+1})=2H_{n}-\frac{n}{n+1}$ \par
(2)令$u(x)=2x+1,\Delta v(x)=\frac{1}{x(x+1)}=(x-1)^{\underline{-2}}$,\par 
所以$\Delta u(x)=u(x+1)-u(x)=2,v(x)=-(x-1)^{\underline{-1}}=-\frac{1}{x},E_{v}(x)=-\frac{1}{x+1}$,\par
所以$\sum (2x+1)\frac{1}{x(x+1)}\delta x=(2x+1)(-\frac{1}{x})-\sum (-\frac{2}{x+1})\delta x=-2-\frac{1}{x}+2H_{x}+C$,\par
所以$\sum_{k=1}^{n}\frac{2k+1}{k(k+1)}$ \par
$= \sum_{1}^{n+1}(2x+1)\frac{1}{x(x+1)}\delta x$ \par
$=\left (-2-\frac{1}{x}+2H_{x}  \right )|_{1}^{n+1}$ \par
$=(-2-\frac{1}{n+1}+2H_{n+1})-(-2-1+2H_{1})=2H_{n}-\frac{n}{n+1}$ \par
~\\
24 令$u(x)=H_{x},\Delta v(x)=\frac{1}{(x+1)(x+2)}=x^{\underline{-2}}$,\par
所以$\Delta u(x)=\frac{1}{x+1},v(x)=-x^{\underline{-1}}=-\frac{1}{x+1}$,\par
所以$E_{v}(x)=-\frac{1}{x+2}$,\par
所以$\sum H_{x}\frac{1}{(x+1)(x+2)}\delta x$\par
$=\sum u_{x}\Delta v_{x}\delta x$\par
$=u_{x}v_{x}-\sum E_{v}(x)\Delta u_{x}\delta  x$\par
$=-\frac{H_{x}}{x+1}-\sum (-\frac{1}{x+2})\frac{1}{x+1}\delta  x$\par
$=-\frac{H_{x}}{x+1}+\sum x^{\underline{-2}}\delta x$\par
$=-\frac{H_{x}}{x+1}-\frac{1}{x+1}$\par
$=-\frac{H_{x}+1}{x+1}$\par
所以$\sum_{0\leq k < n} H_{k}\frac{1}{(k+1)(k+2)}$\par
$=\left (-\frac{H_{x}+1}{x+1}  \right )|_{0}^{n}$\par
$=-\frac{H_{n}+1}{n+1}-(-1)=1-\frac{H_{n}+1}{n+1}$\par
~\\
25、$\sum_{k\in K}ca_{k}=c\sum_{k\in K}a_{k}\leftrightarrow \prod _{k\in K}a_{k}^{c}=\left (\prod _{k\in K}a_{k}  \right )^{c}$ \par
$\sum_{k\in K}(a_{k}+b_{k})=\sum_{k\in K}a_{k}+\sum_{k\in K}b_{k}\leftrightarrow \prod _{k\in K}a_{k}b_{k}=\left (\prod _{k\in K}a_{k}  \right )\left (\prod _{k\in K}b_{k}  \right )$ \par
~\\
26、$P^{2}=\left (\prod_{1\leq j, k\leq n}a_{j}a_{k}  \right )\left ( \prod _{k=1}^{n}a_{k}^{2} \right )$ \par
$=\left (\prod_{1\leq k\leq n}a_{k}^{2n}  \right )\left ( \prod _{k=1}^{n}a_{k}^{2} \right )$\par
$= \prod _{k=1}^{n}a_{k}^{2n+2} \rightarrow P=\prod _{k=1}^{n}a_{k}^{n+1}$\par
~\\
27、$\Delta ((-2)^{\underline{x}})$ \par
$=(-2)^{\underline{x+1}}-(-2)^{\underline{x}}$ \par
$=(-2-x-1)(-2)^{\underline{x}}$ \par
$=\frac{(-2)^{\underline{x}}(-2-x)(-2-(x+1))}{(-2-x)}$ \par
$=-\frac{(-2)^{\underline{x+2}}}{x+2}$ \par
所以$\Delta (-(-2)^{\underline{x-2}})=\frac{(-2)^{\underline{x}}}{x}$ \par
所以$\sum \frac{(-2)^{\underline{x}}}{x}\delta x=-(-2)^{\underline{x-2}}$ \par
所以$\sum_{k=1}^{n} \frac{(-2)^{\underline{k}}}{k}=\left (-(-2)^{\underline{x-2}}  \right  )|_{1}^{n+1}=(-1)^{n}n!-1$ \par
~\\
28、不太清楚。\par
~\\
29 $\sum_{k=1}^{n}\frac{(-1)^{k}k}{4k^{2}-1}$ \par
$=\frac{1}{4}\sum_{k=1}^{n}(-1)^{k}(\frac{1}{2k-1}+\frac{1}{2k+1})$ \par
$=\frac{1}{4}(-\frac{1}{1}-\frac{1}{3}+\frac{1}{3}+\frac{1}{5}-\frac{1}{5}-\frac{1}{7}...+\frac{(-1)^{n}}{2n-1}+\frac{(-1)^{n}}{2n+1})$ \par
$=\frac{1}{4}(-1+\frac{(-1)^{n}}{2n+1})$ \par 
~\\
30 
\begin{itemize}
	\item 设$n$可以表示成$k$个连续正整数之和,最小的正整数为$a \ge  1$
	\item 所以$n=\frac{(a+a+k-1)k}{2}$,所以$1 \le 2a-1=\frac{2n}{k}-k$,即$\frac{2n}{k}-k$是个奇数并且大于等于1
	\item 令$\frac{2n}{k}=X,k=Y$,即$2a-1=X-Y$
\end{itemize}
下面的证明分为两步:
\begin{itemize}
	\item $X,Y$中有且仅有一个是$n$的奇数因子
	\item $n$的每一个奇数因子对应一组答案,即一个二元组$(k,a)$
\end{itemize}
第一条证明:令$n=2^{p}q,k=2^{t}r$,其中$q,r$是奇数\par
\begin{itemize}
	\item 情况1: 如果$k=1$,那么有$Y=1$,是$n$的技术因子,而$X=2n$是偶数
	\item 情况2: 如果$t=p+1,r=1$,那么有$X=q,Y=2^{p+1}$,只有$X$是奇数因子
	\item 情况3: 如果$t=p+1,r > 1$且$r$可以整除$q$,那么有$X=\frac{q}{r},Y=2^{p+1}r$,只有$X$是奇数因子
	\item 情况4: 如果$t=0,r > 1$且$r$可以整除$q$,那么有$X=\frac{2^{p+1}q}{r},Y=r$,只有$Y$是奇数因子
\end{itemize}
第二条证明:\par
\begin{itemize}
	\item $n$的每一个奇数因子,一定对应某个$X$或者$Y$,而不会是$X_{1},Y_{1}$中的$X_{1}$以及$X_{1},Y_{2}$中的$Y_{2}$
	\item 假设存在一个奇数因子$h=X_{1}=Y_{2}$.
	\item 那么$X_{1}-Y_{1}=h-\frac{2n}{h},X_{2}-Y_{2}=\frac{2n}{h}-h$
	\item 它们是相反数,所以只能有一个为正数
	\item 不会有一个奇数因子不对应一组解,因为它不会是第一条证明的四种情况之外的情况
\end{itemize}
下面就是如何计算$n$的奇数因子的个数,由于奇数因子是由若干个奇质数的乘积,所以令$n=2^{e}p_{1}^{t_{1}}p_{2}^{t_{2}}...p_{k}^{t_{k}}$,其中$p_{1},p_{2},...,p_{k}$是奇素数,那么$n$的奇因子的个数为$(t_{1}+1)(t_{2}+1)...(t_{k}+1)$ \par
~\\
~\\
31 $\sum_{k\geq 2}(\zeta (k)-1)$ \par
$=\sum_{t\geq 2}\sum_{k\geq 2}\frac{1}{t^{k}}$ \par
$=\sum_{t\geq 2}\frac{1}{(t-1)t}$ \par
$=\sum_{t\geq 2}(\frac{1}{t-1}-\frac{1}{t})=1$ \par
$\sum_{k\geq 1}(\zeta (2k)-1)$ \par
$=\sum_{t\geq 2}\sum_{k\geq 1}\frac{1}{t^{2k}}$ \par
$=\sum_{t\geq 2}\frac{1}{t^{2}-1}$ \par
$=\sum_{t\geq 2}\frac{1}{2}(\frac{1}{t-1}-\frac{1}{t+1})$ \par
$=\frac{1}{2}(1+\frac{1}{2})=\frac{3}{4}$ \par
~\\
32 分两种情况: \par
(1)$2n\leq x < 2n+1$
\begin{itemize}
	\item 左边=$\sum_{k=0}^{n}k+\sum_{k=n+1}^{2n}(x-k)=\sum_{k=1}^{n}k+\sum_{k=1}^{n}(x-(k+n))=\sum_{k=1}^{n}(x-n)=n(x-n)$
	\item 右边等于$\sum_{k=0}^{n-1}(x-(2k+1))=nx-\sum_{k=0}^{n-1}(2k+1)=nx-n^{2}=n(x-n)$
\end{itemize}


(2)$2n-1\leq x < 2n$
\begin{itemize}
	\item 左边=$\sum_{k=0}^{n-1}k+\sum_{k=n}^{2n-1}(x-k)=\sum_{k=0}^{n-1}k+\sum_{k=0}^{n-1}(x-(k+n))=\sum_{k=0}^{n-1}(x-n)=n(x-n)$
	\item 右边=$\sum_{k=0}^{n-1}(x-(2k+1))=n(x-n)$
\end{itemize}
~\\
\end{document}
