\documentclass[onecolumn]{article}
\usepackage{xeCJK}
\usepackage{amsmath}
\usepackage{listings}
\usepackage{xcolor}
\usepackage{hyperref}
\setlength{\parindent}{0pt}
\renewcommand{\baselinestretch}{1.0}
\lstset{
	frame=tb, % draw a frame at the top and bottom of the code block
	showstringspaces=false, % don't mark spaces in strings
	numbers=left, % display line numbers on the left
	commentstyle=\color{green}, % comment color
	keywordstyle=\color{blue}, % keyword color
	stringstyle=\color{red} % string color
}
\usepackage[a4paper,left=20mm,right=20mm,top=15mm,bottom=15mm]{geometry}  


\begin{document}
1 $11^{4}=(10+1)^{4}=C_{4}^{0}10^{0}+C_{4}^{1}10^{1}+C_{4}^{2}10^{2}+C_{4}^{3}10^{3}+C_{4}^{4}10^{4}=1+40+600+4000+10000=14641$ \par
~\\
2 $\frac{C_{b}^{k+1}}{C_{n}^{k}}=\frac{n-k}{k+1}$.令$\frac{n-k}{k+1}\ge 1$得到$k\le \frac{n+1}{2}$,所以$k \in [1, \frac{n+1}{2})$时,$C_{n}^{k+1}>C_{n}^{k}$.同理在大于$\frac{n}{2}$的时候逐渐减小。所以在中间位置最大。所以结论为:$n$为偶数时,最大值为$k=\frac{n}{2}$;奇数时,在$k=\frac{n-1}{2}$和$k=\frac{n+1}{2}$时最大 \par 
~\\
3 $\binom{n-1}{k-1}/\binom{n-1}{k}=\frac{k}{n-k}$\par
$\binom{n}{k+1}/\binom{n}{k-1}=\frac{(n-k+1)(n-k)}{k(k+1)}$ \par
$\binom{n+1}{k}/\binom{n+1}{k+1}=\frac{k+1}{n+1-k}$ \par
相乘得1.随意相等 \par
~\\
4 当$k \ge 0$时,由公式5.14,令$r=-1$得到$\binom{-1}{k}=(-1)^{k}\binom{k-(-1)-1}{k}=(-1)^{k}$ ;$k<0$时结果为0 \par
~\\
5 $\binom{p}{k}=\frac{p(p-1)(p-2)..(p-k+1)}{k!}$是个整数。但是$k!$不能整除$p$,所以$p$可以整除$\binom{p}{k}$\par
令$a=\binom{p-1}{k}=\frac{(p-1)(p-2)...(p-k)}{k!}$,所以$a*k!=(p-1)(p-2)..(p-k)$,两边模$p$得到$a*k!\equiv (-1)^{k}k!(\mod p)$,由于$k!$不能整除$p$,所以$a\mod p=(-1)^{k}$ \par
~\\
6 $\sum_{k\ge 0}\binom{n+k}{2k}\binom{2k}{k}\frac{(-1)^{k}}{k+1}$ \par
$=\sum_{k\ge 0}\binom{n+k}{k}\binom{n}{k}\frac{(-1)^{k}}{k+1}$(根据公式5.21)  \par
由于$\binom{n+1}{k+1}=\frac{n+1}{k+1}\binom{n}{k}$,所以$\binom{n}{k}\frac{1}{k+1}=\binom{n+1}{k+1}\frac{1}{n+1}$ \par
那么上面的式子$=\frac{1}{n+1}\sum_{k\ge 0}\binom{n+k}{k}\binom{n+1}{k+1}(-1)^{k}$ \par
$=\frac{1}{n+1}\sum_{k\ge 0}\binom{n+k}{n}\binom{n+1}{k+1}(-1)^{k}$ \par
$=\frac{1}{n+1}\sum_{k}\binom{n+k}{n}\binom{n+1}{k+1}(-1)^{k}-\frac{1}{n+1}\binom{n-1}{n}\binom{n+1}{0}(-1)^{-1}$ (后面这一部分是$k=-1$,$k<-1$时都是0)\par
$\frac{1}{n+1}\binom{n-1}{n}\binom{n+1}{0}(-1)^{-1}$仅当$n=0$时为-1,其他值结果都是0 \par
对于$\frac{1}{n+1}\sum_{k}\binom{n+k}{n}\binom{n+1}{k+1}(-1)^{k}$可利用公式5.24,$(s,n,l,m)\rightarrow(n,n,n+1,1)$得到 \par
$\frac{1}{n+1}\sum_{k}\binom{n+k}{n}\binom{n+1}{k+1}(-1)^{k}=\frac{1}{n+1}(-1)^{n+2}\binom{n-1}{-1}=0$ \par
所以最后的答案是$[n=0]$ \par
~\\ 
7 当$k>0$时,$x^{\underline{-k}}=\frac{(-1)^{k}}{(-x-1)^{\underline{k}}}$ \par
所以当$k>0$时,$r^{\underline{-k}}(r-\frac{1}{2})^{\underline{-k}}$
$=\frac{(-1)^{k}}{(-r-1)(-r-2)..(-r-k)}\frac{(-1)^{k}}{(-r-\frac{1}{2})(-r-\frac{3}{2})..(-r-k+\frac{1}{2})}$ \par
$=\frac{2^{2k}}{(-2r-1)(-2r-2)(-2r-3)...(-2r-2k)}$ \par
同理,右侧$=\frac{(-1)^{2k}}{(-2r-1)^{2k}}\frac{1}{2^{-2k}}$ \par
=$=\frac{2^{2k}}{(-2r-1)(-2r-2)...(-2r-2k)}$ \par
因此,$k<0$时仍成立 \par
~\\
8 $\sum_{k}\binom{n}{k}(-1)^{k}(1-\frac{k}{n})^{n}$ \par
$=\sum_{k}\binom{n}{k}(-1)^{-k}(1-\frac{k}{n})^{n}$ \par
$=\sum_{k}\binom{n}{k}(-1)^{-k}(-1)^{n}(\frac{k}{n}-1)^{n}$ \par
$=\sum_{k}\binom{n}{k}(-1)^{n-k}(\frac{k}{n}-1)^{n}$ \par
令$f(k)=(\frac{k}{n}-1)^{n}$并在公式5.40中令$x=0$可以得到: \par
$\Delta^{n}f(0)=\sum_{k}\binom{n}{k}(-1)^{n-k}f(k)=\sum_{k}\binom{n}{k}(-1)^{n-k}(\frac{k}{n}-1)^{n}$ \par
所以结果就是$\Delta^{n}f(0)$ \par
将$f(x)=(\frac{x}{n}-1)^{n}$展开成$f(x)=\sum_{d=0}^{d=n}a_{d}x^{d}$所以$a_{n}=n^{-n}$,再换算成牛顿级数的系数$c_{n}=n^{-n}n!$,所以$\Delta^{n}f(0)=c_{n}=n^{-n}n!$ \par
$n$无穷大时,根据\href{https://baike.baidu.com/item/%E6%96%AF%E7%89%B9%E6%9E%97%E5%85%AC%E5%BC%8F/9583086}{斯特林近似},得到$\frac{n!}{n^{n}}\approx \frac{\sqrt{2 \pi n}}{e^{n}}$ \par
~\\
9 (超几何函数相关)不知道\par
~\\
10 (超几何函数相关)不知道\par
~\\
11 (超几何函数相关)不知道\par
~\\
12 (超几何函数相关)不知道\par
~\\
13 $P_{n}$中数字$k\in [1,n]$出现了$n+1-k$次 \par
$Q_{n}$中数字$k\in [1,n]$出现了$k$次 \par
将$\binom{n}{k}=\frac{n!}{k!(n-k)!}$代入$R_{n}$得到$R_{n}=\frac{(n!)^{n+1}}{(1!2!3!...n!)^{2}}$ \par
所以分母是$P_{n}^{2}$,分子是$P_{n}Q_{n}$。因此$R_{n}=\frac{Q_{n}}{P_{n}}$ \par
~\\
14 $\sum_{k \le l}\binom{l-k}{m}\binom{s}{k-n}(-1)^{k}$ \par
$=\sum_{k \le l}\binom{l-k}{l-k-m}\binom{s}{k-n}(-1)^{k}$ (这个地方$m>l-k$的项都是0)\par
$=\sum_{k \le l}(-1)^{l-k-m}\binom{-m-1}{l-k-m}\binom{s}{k-n}(-1)^{k}$ \par
$=\sum_{k \le l}\binom{-m-1}{l-k-m}\binom{s}{k-n}(-1)^{l-m}$ \par
$=\sum_{k}\binom{-m-1}{l-k-m}\binom{s}{k-n}(-1)^{l-m}$ ($k>l$的项都是0, $(-1)^{l-m}$是常数)\par
$=(-1)^{l-m}\binom{s-m-1}{l-m-n}$ \par
$\sum_{-q \le k \le l}\binom{l-k}{m}\binom{q+k}{n}$\par
$=\sum_{-q \le k \le l}\binom{l-k}{l-k-m}\binom{q+k}{q+k-n}$ \par
$=\sum_{-q \le k \le l}(-1)^{l-k-m}\binom{-m-1}{l-k-m}(-1)^{q+k-n}\binom{-n-1}{q+k-n}$ \par
$=\sum_{-q \le k \le l}\binom{-m-1}{l-k-m}\binom{-n-1}{q+k-n}(-1)^{l+q-m-n}$ \par
$=\sum_{k}\binom{-m-1}{l-k-m}\binom{-n-1}{q+k-n}(-1)^{l+q-m-n}$ ($k>l$时前面一项为0,$k<-q$时后面一项为0)\par
$=\binom{-m-n-2}{l+q-m-n}(-1)^{l+q-m-n}$ \par
$=\binom{l+q+1}{l+q-m-n}$ \par
$=\binom{l+q+1}{m+n+1}$ \par
~\\
15 (1)若$n$是奇数,那么$k$和$n-k$必是一个奇数一个偶数\par
那么$\binom{n}{k}^{3}(-1)^{k}+\binom{n}{n-k}^{3}(-1)^{n-k}$ \par
$=\binom{n}{k}^{3}((-1)^{k}+(-1)^{n-k})=0$ \par
而$\binom{n}{k}$一共有偶数项,$[0,1,..,n]$,每两项得到0,所以和是0\par
(2)若$n$时偶数,令$n=2m$,在公式5.29中,令$a=b=c=m$得到$\sum_{k}\binom{2m}{m+k}^{3}(-1)^{k}=\frac{(3m)!}{(m!)^{3}}$ \par
令$k=-m+k$得到$\sum_{-m+k}\binom{2m}{k}^{3}(-1)^{-m+k}=\frac{(3m)!}{(m!)^{3}}$ \par
两边同时乘以$(-1)^{m}$得到$\sum_{-m+k}\binom{2m}{k}^{3}(-1)^{k}=(-1)^{m}\frac{(3m)!}{(m!)^{3}}$ \par
左侧$-m+k$取任意值,所以$\sum_{k}\binom{2m}{k}^{3}(-1)^{k}=(-1)^{m}\frac{(3m)!}{(m!)^{3}}$ \par
~\\
16 $\binom{2a}{a+k}\binom{2b}{(b+k)}\binom{2c}{c+k}$ \par
$=\frac{(2a)!(2b)!(2c)!}{(a+k)!(a-k)!(b+k)!(b-k)!(c+k)!(c-k)!}$ \par
$=\frac{(2a)!(2b)!(2c)!}{(a+b)!(b+c)!(c+a)!}\frac{(a+b)!(b+c)!(c+a)!}{(a+k)!(a-k)!(b+k)!(b-k)!(c+k)!(c-k)!}$ \par
$=\frac{(2a)!(2b)!(2c)!}{(a+b)!(b+c)!(c+a)!}\binom{a+b}{a+k}\binom{b+c}{b+k}\binom{c+a}{c+k}$ \par
所以由公式5.29得到$\sum_{k}\binom{2a}{a+k}\binom{2b}{(b+k)}\binom{2c}{c+k}(-1)^{k}$ \par
$=\frac{(2a)!(2b)!(2c)!}{(a+b)!(b+c)!(c+a)!}\frac{(a+b+c)!}{a!b!c!}$   \par
~\\
\end{document}
